\documentclass[11pt]{article}
\textwidth 15cm 
\textheight 21.3cm
\evensidemargin 6mm
\oddsidemargin 6mm
\topmargin -1.1cm
\setlength{\parskip}{1.5ex}


\usepackage{amsfonts,amsmath,amssymb,enumerate}



\begin{document}
\textbf{Question 1} [15 marks] \\\\
\textbf{a)} Let u and v be vectors in $\mathbb{R}^n$, we will start by assuming that $(u \bullet v) = 0$, it thus follows that:
\begin{align*}
(u \bullet v)  &= 0 \\
(u \bullet v)  + (u \bullet v) &= 0 + 0\\
(u \bullet v) + (v \bullet u) &= 0
\end{align*}
We will now add ($u \bullet u$) and ($v \bullet v$) to both sides:
\begin{align*}
(u \bullet u) + (u \bullet v)  + (v \bullet u) + (v \bullet v)  &=(u \bullet u) + 0 + (v \bullet v)\\
(u \bullet u) + (u \bullet v)  + (v \bullet u) + (v \bullet v)  &=(u \bullet u) + (v \bullet v)
\end{align*}
We know from the properties of the dot product that $(u\bullet u) = ||u||^2$, so our equation becomes:
\begin{align*}
(u \bullet u) + (u \bullet v)  + (v \bullet u) + (v \bullet v)  &=||u||^2 + (v \bullet v)\\
(u \bullet u) + (u \bullet v)  + (v \bullet u) + (v \bullet v)  &=||u||^2 + ||v||^2
\end{align*}
We know from the proprieties of the dot product that $(u \bullet v) + (u \bullet u) = u \bullet (u+ v)$, so we will distrabute out $u$ and then $v$ to get:
\begin{align*}
u \bullet (u \bullet v)  + (v \bullet u) + (v \bullet v)  &=||u||^2 + ||v||^2\\
u \bullet (u + v) +  v \bullet (u + v) &=||u||^2 + ||v||^2
\end{align*}
As well since $(u + v) \bullet (u + v)  = u \bullet (u + v) +  v \bullet (u + v)$, it implies that our equation will become:
\begin{align*}
(u + v) \bullet (u + v) &=||u||^2 + ||v||^2 \\
||u + v||^2 &=||u||^2 + ||v||^2
\end{align*}\\\\
\textbf{b)} The converse of this statement would be (for vectors u and v in $\mathbb{R}^n$)
\[ \text{if } ||u + v||^2 = ||u||^2 + ||v||^2 \text{ then } (u \bullet v) = 0 \]\\\\
\textbf{c)} To start we will assume the hypothesis (for vectors u and v in $\mathbb{R}^n$):
\[ \text{let } ||u+v||^2 = ||u||^2 + ||v||^2 \]
By the definition of length for a vector, it imples that the equation becomes: 
\begin{align*}
||u + v||^2 &=||u||^2 + ||v||^2 \\
(\sqrt{(u + v) \bullet (u + v)})^2 &= (\sqrt{(u \bullet u)})^2 + (\sqrt{(v \bullet v)})^2 \\
(u + v) \bullet (u + v) &= (\sqrt{(u \bullet u)})^2 + (\sqrt{(v \bullet v)})^2 \\
(u + v) \bullet (u + v) &= (u \bullet u) + (v \bullet v) 
\end{align*}
We know by definition that $(u + v) \bullet (u + v)  = u \bullet (u + v) +  v \bullet (u + v)$, thus:
\begin{align*}
(u + v) \bullet (u + v) &= (u \bullet u) + (v \bullet v) \\
u \bullet (u + v) +  v \bullet (u + v)   &= (u \bullet u) + (v \bullet v)  
\end{align*}
Futhermore we know that $(u \bullet v) + (u \bullet u) = u \bullet (u+ v)$, which thus implies:
\begin{align*}
u \bullet (u + v) +  v \bullet (u + v)   &= (u \bullet u) + (v \bullet v) \\
(u \bullet u) + (u \bullet v)  + (v \bullet u) + (v \bullet v)   &= (u \bullet u) + (v \bullet v) \\
(u \bullet v)  + (v \bullet u) &= (u \bullet u) + (v \bullet v) -   (v \bullet v)  - (u \bullet u)\\
(u \bullet v)  + (v \bullet u) &= 0
\end{align*}
From the properties of dot product we know this can simplify to:
\begin{align*}
(u \bullet v)  + (v \bullet u) &= 0\\
(u \bullet v)  + (u \bullet v) &= 0\\
2(u \bullet v) &= 0\\
(u \bullet v) &= 0
\end{align*}
Therefore proving that for all vectors $u$, $v$ in  $\mathbb{R}^n$, that if $||u + v||^2 =||u||^2 + ||v||^2$ then $(u \bullet v) = 0$\\\\
\textbf{d)}
\end{document}