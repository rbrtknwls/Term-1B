\documentclass[11pt]{article}
\textwidth 15cm 
\textheight 21.3cm
\evensidemargin 6mm
\oddsidemargin 6mm
\topmargin -1.1cm
\setlength{\parskip}{1.5ex}


\usepackage{amsfonts,amsmath,amssymb,enumerate}



\begin{document}
\textbf{Question 1} [15 marks] \\\\
\textbf{a)} Let u and v be vectors in $\mathbb{R}^n$, we will start by assuming that $(u \bullet v) = 0$, it thus follows that:
\begin{align*}
(u \bullet v)  &= 0 \\
(u \bullet v)  + (u \bullet v) &= 0 + 0\\
(u \bullet v) + (v \bullet u) &= 0
\end{align*}
We will now add ($u \bullet u$) and ($v \bullet v$) to both sides:
\begin{align*}
(u \bullet u) + (u \bullet v)  + (v \bullet u) + (v \bullet v)  &=(u \bullet u) + 0 + (v \bullet v)\\
(u \bullet u) + (u \bullet v)  + (v \bullet u) + (v \bullet v)  &=(u \bullet u) + (v \bullet v)
\end{align*}
We know from the properties of the dot product that $(u\bullet u) = ||u||^2$, so our equation becomes:
\begin{align*}
(u \bullet u) + (u \bullet v)  + (v \bullet u) + (v \bullet v)  &=||u||^2 + (v \bullet v)\\
(u \bullet u) + (u \bullet v)  + (v \bullet u) + (v \bullet v)  &=||u||^2 + ||v||^2
\end{align*}
We know from the proprieties of the dot product that $(u \bullet v) + (u \bullet u) = u \bullet (u+ v)$, so we will distribute out $u$ and then $v$ to get:
\begin{align*}
u \bullet (u \bullet v)  + (v \bullet u) + (v \bullet v)  &=||u||^2 + ||v||^2\\
u \bullet (u + v) +  v \bullet (u + v) &=||u||^2 + ||v||^2
\end{align*}
As well since $(u + v) \bullet (u + v)  = u \bullet (u + v) +  v \bullet (u + v)$, it implies that our equation will become:
\begin{align*}
(u + v) \bullet (u + v) &=||u||^2 + ||v||^2 \\
||u + v||^2 &=||u||^2 + ||v||^2
\end{align*}\\\\
\textbf{b)} The converse of this statement would be (for vectors u and v in $\mathbb{R}^n$)
\[ \text{if } ||u + v||^2 = ||u||^2 + ||v||^2 \text{ then } (u \bullet v) = 0 \]\\\\
\textbf{c)} To start we will assume the hypothesis (for vectors u and v in $\mathbb{R}^n$):
\[ \text{let } ||u+v||^2 = ||u||^2 + ||v||^2 \]
By the definition of length for a vector, it implies that the equation becomes: 
\begin{align*}
||u + v||^2 &=||u||^2 + ||v||^2 \\
(\sqrt{(u + v) \bullet (u + v)})^2 &= (\sqrt{(u \bullet u)})^2 + (\sqrt{(v \bullet v)})^2 \\
(u + v) \bullet (u + v) &= (\sqrt{(u \bullet u)})^2 + (\sqrt{(v \bullet v)})^2 \\
(u + v) \bullet (u + v) &= (u \bullet u) + (v \bullet v) 
\end{align*}
We know by definition that $(u + v) \bullet (u + v)  = u \bullet (u + v) +  v \bullet (u + v)$, thus:
\begin{align*}
(u + v) \bullet (u + v) &= (u \bullet u) + (v \bullet v) \\
u \bullet (u + v) +  v \bullet (u + v)   &= (u \bullet u) + (v \bullet v)  
\end{align*}
Furthermore we know that $(u \bullet v) + (u \bullet u) = u \bullet (u+ v)$, which thus implies:
\begin{align*}
u \bullet (u + v) +  v \bullet (u + v)   &= (u \bullet u) + (v \bullet v) \\
(u \bullet u) + (u \bullet v)  + (v \bullet u) + (v \bullet v)   &= (u \bullet u) + (v \bullet v) \\
(u \bullet v)  + (v \bullet u) &= (u \bullet u) + (v \bullet v) -   (v \bullet v)  - (u \bullet u)\\
(u \bullet v)  + (v \bullet u) &= 0
\end{align*}
From the properties of dot product we know this can simplify to:
\begin{align*}
(u \bullet v)  + (v \bullet u) &= 0\\
(u \bullet v)  + (u \bullet v) &= 0\\
2(u \bullet v) &= 0\\
(u \bullet v) &= 0
\end{align*}
Therefore proving that for all vectors $u$, $v$ in  $\mathbb{R}^n$, that if $||u + v||^2 =||u||^2 + ||v||^2$ then $(u \bullet v) = 0$\\\\
\textbf{d)} For vectors $u,v,w \in \mathbb{R}^n$, the statement can be extended to:
\[ \text{if } (u \bullet v) = 0, (u \bullet w) = 0 \text{ and } (v \bullet w) \text{ then } || u + v + w ||^2 = ||u||^2 + ||v||^2 + ||w||^2\]
If we assume the hypothesis is correct we get the equation we get (and it thus follows that):
\begin{align*}
(u \bullet v)  + (w \bullet u) + (v \bullet w) &= 0 + 0 + 0\\
(u \bullet v)  + (w \bullet u) + (v \bullet w)  &= 0 \\
2(u \bullet v)  + 2(w \bullet u) + 2(v \bullet w)  &= 0
\end{align*}
If we then add ($u \bullet u$), ($v \bullet v$) and ($w \bullet w$) to both sides and apply the definition of length of a vector we thus get:
\begin{align*}
2(u \bullet v)  + 2(w \bullet u) + 2(v \bullet w) + (u \bullet u) + (v \bullet v) + (w \bullet w) &= 0 + (u \bullet u) + (v \bullet v) + (w \bullet w)\\
2(u \bullet v)  + 2(w \bullet u) + 2(v \bullet w) + (u \bullet u) + (v \bullet v) + (w \bullet w) &= ||u||^2 + (v \bullet v) + (w \bullet w)\\
2(u \bullet v)  + 2(w \bullet u) + 2(v \bullet w) + (u \bullet u) + (v \bullet v) + (w \bullet w) &= ||u||^2 + ||v||^2 + (w \bullet w)\\
2(u \bullet v)  + 2(w \bullet u) + 2(v \bullet w) + (u \bullet u) + (v \bullet v) + (w \bullet w) &= ||u||^2 + ||v||^2 + ||w||^2\\
\end{align*}
From the definition we know that $(u \bullet v) + (u \bullet u) + (u \bullet w)  = u \bullet ( u+ v + w)$, so by arragning we get that:
\begin{align*}
u \bullet (u+v +w)  + (u \bullet v)  + (w \bullet u) + 2(v \bullet w) + (v \bullet v) + (w \bullet w) &= ||u||^2 + ||v||^2 + ||w||^2\\
u \bullet (u+v +w)  + v \bullet(v + u + w)  + (w \bullet u) + (v \bullet w) + (w \bullet w) &= ||u||^2 + ||v||^2 + ||w||^2\\
u \bullet (u+v +w)  + v \bullet(u + v + w)  + w \bullet (u + v + w)  &= ||u||^2 + ||v||^2 + ||w||^2\\
\end{align*}
We know that $(u+v+w) \bullet (u + v + w) = u \bullet (u+v +w)  + v \bullet(u + v + w)  + w \bullet (u + v + w)$ and by the definition of length this becomes:
\begin{align*}
u \bullet (u+v +w)  + v \bullet(u + v + w)  + w \bullet (u + v + w)  &= ||u||^2 + ||v||^2 + ||w||^2\\
(u+v+w) \bullet (u+v +w)   &= ||u||^2 + ||v||^2 + ||w||^2\\
||u+v +w||^2   &= ||u||^2 + ||v||^2 + ||w||^2
\end{align*}
This thus proves for all vectors $u,v,w \in \mathbb{R}^n$ if $(u \bullet v) = 0, (u \bullet w) = 0 \text{ and } (v \bullet w)$, then:
\[||u+v +w||^2   = ||u||^2 + ||v||^2 + ||w||^2\]\\\\
\textbf{e)} The extension to b) for vectors $u,v,w$ in $\mathbb{R}^2$ can be written as:
\[ \text{if } ||u + v + w||^2 = ||u||^2 + ||v||^2 + ||w||^2 \text{ then } (u \bullet v) = 0, (y \bullet w) = 0 \text{ and } (y \bullet w) = 0 \]\\\\
\text{f)} For the counter example let the following vectors (in $\mathbb{R}^3$) equal:
\begin{equation*}
u = 
\begin{pmatrix}
-1/2 \\
-2\\
-2
\end{pmatrix}
\text{  }v =
\begin{pmatrix}
-1 \\
-1\\
-1
\end{pmatrix}
\text{  } w =
\begin{pmatrix}
-1 \\
1\\
1
\end{pmatrix}
\end{equation*}
To start we will check if the hypothesis holds, thus we will check if:
\begin{align*}
||u+v +w||^2   &= ||u||^2 + ||v||^2 + ||w||^2 \\
||u+v +w||^2 &=  (u \bullet u) + (v \bullet v) + (w \bullet w)\\
(u+v+w) \bullet (u+v +w) &=  (u \bullet u) + (v \bullet v) + (w \bullet w)\\
\end{align*}
Plugging our values our equation becomes:
\begin{equation*}
\begin{pmatrix}
-1/2 + -1 - 1\\
-2 -1 + 1\\
-2 -1 + 1
\end{pmatrix}
\bullet
\begin{pmatrix}
-1/2 -1 - 1\\
-2 -1 + 1\\
-2 -1 + 1
\end{pmatrix}
=
\begin{pmatrix}
-1/2 \\
-2\\
-2
\end{pmatrix}
\bullet
\begin{pmatrix}
-1/2 \\
-2\\
-2
\end{pmatrix}
+
\begin{pmatrix}
-1 \\
-1\\
-1
\end{pmatrix}
\bullet
\begin{pmatrix}
-1 \\
-1\\
-1
\end{pmatrix}
+
\begin{pmatrix}
-1 \\
1\\
1
\end{pmatrix}
\bullet
\begin{pmatrix}
-1 \\
1\\
1
\end{pmatrix}
\end{equation*}
\begin{equation*}
\begin{pmatrix}
-5/2\\
-2\\
-2
\end{pmatrix}
\bullet
\begin{pmatrix}
-5/2\\
-2\\
-2
\end{pmatrix}
=
\begin{pmatrix}
-1/2 \\
-2\\
-2
\end{pmatrix}
\bullet
\begin{pmatrix}
-1/2 \\
-2\\
-2
\end{pmatrix}
+
\begin{pmatrix}
-1 \\
-1\\
-1
\end{pmatrix}
\bullet
\begin{pmatrix}
-1 \\
-1\\
-1
\end{pmatrix}
+
\begin{pmatrix}
-1 \\
1\\
1
\end{pmatrix}
\bullet
\begin{pmatrix}
-1 \\
1\\
1
\end{pmatrix}
\end{equation*}
\begin{equation*}
\begin{pmatrix}
-5/2\\
-2\\
-2
\end{pmatrix}
\bullet
\begin{pmatrix}
-5/2\\
-2\\
-2
\end{pmatrix}
=
\begin{pmatrix}
-1/2 \\
-2\\
-2
\end{pmatrix}
\bullet
\begin{pmatrix}
-1/2 \\
-2\\
-2
\end{pmatrix}
+
\begin{pmatrix}
-1 \\
-1\\
-1
\end{pmatrix}
\bullet
\begin{pmatrix}
-1 \\
-1\\
-1
\end{pmatrix}
+ (-1)*(-1) + 1*1 + 1*1
\end{equation*}
\begin{equation*}
\begin{pmatrix}
-5/2\\
-2\\
-2
\end{pmatrix}
\bullet
\begin{pmatrix}
-5/2\\
-2\\
-2
\end{pmatrix}
=
\begin{pmatrix}
-1/2 \\
-2\\
-2
\end{pmatrix}
\bullet
\begin{pmatrix}
-1/2 \\
-2\\
-2
\end{pmatrix}
+ (-1)(-1) + (-1)(-1) + (-1)(-1) + 3
\end{equation*}
\begin{equation*}
\begin{pmatrix}
-5/2\\
-2\\
-2
\end{pmatrix}
\bullet
\begin{pmatrix}
-5/2\\
-2\\
-2
\end{pmatrix}
= (-1/2)(-1/2) + (-2)(-2) + (-2)(-2) + 6
\end{equation*}
\begin{align*}
(-\frac{5}{2})(-\frac{5}{2}) + (-2)(-2) + (-2)(-2) &= \frac{57}{4} \\
\frac{25}{4} + 4 + 4 &= \frac{57}{4} \\
\frac{57}{4} &= \frac{57}{4}
\end{align*}
Thus we have proved the hypothesis, however note that the conclusion does not hold as:
\begin{equation*}
u \bullet v =
\begin{pmatrix}
-1/2\\
-2\\
-2
\end{pmatrix}
\bullet
\begin{pmatrix}
-1\\
-1\\
-1
\end{pmatrix}
\end{equation*}
\begin{align*}
u \bullet v &= (-\frac{1}{2})(-1) + (-2)(-1) + (-2)(-1) \\
u \bullet v &= \frac{1}{2} + 2 + 2 \\
u \bullet v &= \frac{9}{2} 
\end{align*}
Because $(u \bullet v) \neq 0$, this shows that the counter examples disproves the origional statement.
\end{document} 