\documentclass[11pt]{article}
\textwidth 15cm 
\textheight 21.3cm
\evensidemargin 6mm
\oddsidemargin 6mm
\topmargin -1.1cm
\setlength{\parskip}{1.5ex}


\usepackage{amsfonts,amsmath,amssymb,enumerate}



\begin{document}
\textbf{Question 1} [15 marks] \\\\
a) We will start by assuming that $(u \bullet v) = 0$, it thus follows that:
\begin{align*}
(u \bullet v)  &= 0 \\
(u \bullet v)  + (u \bullet v) &= 0 + 0\\
(u \bullet v) + (v \bullet u) &= 0
\end{align*}
We will now add ($u \bullet u$) and ($v \bullet v$) to both sides:
\begin{align*}
(u \bullet u) + (u \bullet v)  + (v \bullet u) + (v \bullet v)  &=(u \bullet u) + 0 + (v \bullet v)\\
(u \bullet u) + (u \bullet v)  + (v \bullet u) + (v \bullet v)  &=(u \bullet u) + (v \bullet v)
\end{align*}
We know from the properties of the dot product that $(u\bullet u) = ||u||^2$, so our equation becomes:
\begin{align*}
(u \bullet u) + (u \bullet v)  + (v \bullet u) + (v \bullet v)  &=||u||^2 + (v \bullet v)\\
(u \bullet u) + (u \bullet v)  + (v \bullet u) + (v \bullet v)  &=||u||^2 + ||v||^2
\end{align*}
We know from the proprieties of the dot product that $(u \bullet v) + (u \bullet u) = u \bullet (u+ v)$, so we will distrabute out $u$ and then $v$ to get:
\begin{align*}
u \bullet (u \bullet v)  + (v \bullet u) + (v \bullet v)  &=||u||^2 + ||v||^2\\
u \bullet (u + v) +  v \bullet (u + v) &=||u||^2 + ||v||^2
\end{align*}
As well since $(u + v) \bullet (u + v)  = u \bullet (u + v) +  v \bullet (u + v)$, it implies that our equation will become:
\begin{align*}
(u + v) \bullet (u + v) &=||u||^2 + ||v||^2 \\
||u + v||^2 &=||u||^2 + ||v||^2
\end{align*}\\\\
b) The converse of this statement would be:
\[ \text{if } ||u + v||^2 = ||u||^2 + ||v||^2 \text{ then } (u \bullet v) = 0 \]\\\\
c) To start we know that with three vectors $u, v, w$ the equation would become:
\[ \text{if } u \bullet v \bullet w = 0 \text{ then } ||u + v + w|| = ||u||^2 + ||v||^2 + ||w||^2 \]
It thus follows that:
\begin{align*}
(u \bullet v \bullet w)  &= 0 \\
(u \bullet v \bullet w) + (u \bullet v \bullet w) +(u \bullet v \bullet w)  &= 0 + 0 + 0 \\
(u \bullet (v \bullet w)) + (w \bullet (u \bullet v)) +(v \bullet (u \bullet w))  &= 0 \\
(u \bullet v) + (v \bullet w)+ (w \bullet u) + (w \bullet v) + (v \bullet u) + (v \bullet w)  &= 0 
\end{align*}
By the dot product's property of symmetry we know that this becomes:
\end{document}